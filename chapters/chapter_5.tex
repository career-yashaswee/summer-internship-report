\section{Chapter 5 - AI Integration in the Project}

\subsection{Introduction to AI in Web Development}
Artificial Intelligence \gls{AI} has become a transformative force in web development, enabling the creation of smarter, more responsive applications. As defined by Russell and Norvig (2016), "AI is the science of making machines do things that would require intelligence if done by men" \cite{russell2016artificial}. In the context of web development, AI facilitates tasks such as natural language processing, image recognition, and predictive analytics, significantly enhancing user experience and functionality.

\subsection{Tools and Libraries Used for AI}
The project utilizes several key tools and libraries for AI integration:

\begin{itemize}
    \item Google Cloud: Provides a suite of AI and machine learning services, including Google Cloud Storage for resume uploading and saving, and Google Cloud AI tools for various AI functionalities.
    \item Gemini: Used for Generative Artificial Intelligence, enabling advanced content generation and creative applications.
    \item Google Text-to-Speech (TTS) and Speech-to-Text (STT): These services facilitate voice interaction capabilities, converting spoken language into text and vice versa.
    \item Mongoose: A MongoDB object modeling tool designed to work in an asynchronous environment, facilitating easy interaction with MongoDB databases.
    \item Node.js Modules: Including `ask`, `interpret`, `tts`, `stt`, and `process`, which are used for handling AI-related tasks and processing.
\end{itemize}

\subsection{Implementing AI Features}
AI features were implemented using a combination of Google Cloud services and Node.js modules:

\begin{itemize}
    \item Voice Interaction: Implemented using Google TTS and STT, allowing users to interact with the application via voice commands.
    \item Resume Handling: Utilized Google Cloud Storage Buckets for secure uploading and storage of resumes.
    \item Model Fine-Tuning: The AI model was fine-tuned on Google Cloud Studio Console using predefined datasets from the [Large QA Datasets](https://github.com/ad-freiburg/large-qa-datasets) repository. This process involved training the model on a dataset of questions and ideal answers to improve its accuracy and relevance.
    \item Environment Security: Secured credentials by storing them in a `.env` file, ensuring sensitive information remains protected.
\end{itemize}

\subsection{Challenges and Solutions}
Several challenges were encountered during AI integration, along with their solutions:

\begin{itemize}
    \item Model Accuracy: Ensuring the AI model's accuracy was a challenge. Fine-tuning the model on specific datasets, as described, helped address this issue.
    \item Data Security: Managing sensitive data, such as user resumes and credentials, required robust security measures. Using Google Cloud Storage with secure access and storing credentials in a `.env` file provided a secure solution.
    \item Integration Complexity: Integrating multiple AI services and libraries involved complex configurations. Leveraging Google Cloud's extensive documentation and community support helped navigate these complexities.
\end{itemize}

\subsection{Testing and Optimization}
Testing and optimizing the AI features were crucial for ensuring their functionality and performance:

\begin{itemize}
    \item Testing: Conducted extensive testing to validate the accuracy of voice interactions, resume handling, and overall AI performance. Automated tests were implemented to ensure continuous quality assurance.
    \item Optimization: Regular optimization of AI models and services was performed to enhance performance and reduce latency. This included monitoring system metrics and making adjustments based on user feedback and performance data.
\end{itemize}

\begin{figure}[h]
    \centering
    \includegraphics[width=1.0cm]{ai_integration_diagram.png}
    \caption{Diagram of AI Integration in the Project}
    \label{fig:ai_integration}
\end{figure}

\begin{table}[h]
    \centering
    \begin{tabular}{|l|l|}
        \hline
        \textbf{Tool/Library} & \textbf{Function} \\
        \hline
        Google Cloud & AI and machine learning services, storage \\
        Gemini & Generative AI \\
        Google TTS & Voice interaction, text-to-speech \\
        Google STT & Voice interaction, speech-to-text \\
        Mongoose & MongoDB object modeling \\
        Node.js Modules & Handling AI tasks and processing \\
        \hline
    \end{tabular}
    \caption{Tools and Libraries Used for AI Integration}
    \label{tab:ai_tools}
\end{table}