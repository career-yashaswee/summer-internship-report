\section{Chapter 4 - MERN Stack Implementation}
\subsection{Introduction to MERN Stack}

The \textbf{MERN stack} is a powerful combination of technologies for building modern web applications. It comprises \textbf{MongoDB}, \textbf{Express.js}, \textbf{React.js}, and \textbf{Node.js}, providing a full-stack solution for developing robust, scalable applications. This stack leverages JavaScript throughout, from the client-side to the server-side, facilitating seamless integration and rapid development. The \textbf{MERN stack} is particularly suited for creating single-page applications with a dynamic user interface, offering a cohesive and efficient development experience.

\subsection{Backend Development with Node.js and Express.js}

The backend of our project is built using \textbf{Node.js} and \textbf{Express.js}. \textbf{Node.js} is a powerful JavaScript runtime that allows for asynchronous processing and efficient handling of multiple requests. It is particularly well-suited for applications that require real-time capabilities and scalable performance. \textbf{Express.js} is a lightweight web application framework that sits on top of Node.js, simplifying the process of building and managing RESTful APIs.

In our implementation, we utilized \textbf{Express.js} to design a robust REST API for managing user interactions and data processing. The API endpoints were thoroughly tested using \textbf{Postman}, ensuring that they functioned as expected and adhered to our design specifications. For session management and security, we employed \textbf{JWT (JSON Web Tokens)} for issuing and verifying tokens, along with cookies for secure session management. This approach ensures that user sessions are protected against unauthorized access and tampering.

We used several key dependencies to enhance the functionality and security of our backend:

\begin{table}[h!]
\centering
\begin{tabular}{|l|l|}
\hline
\textbf{Dependency} & \textbf{Description} \\
\hline
\textbf{@google-cloud} & These Google Cloud libraries are essential for integrating advanced AI capabilities such as speech recognition, text-to-speech synthesis, and storage management. \\
\hline
\textbf{dotenv} & Used for managing environment variables, it securely stores sensitive information such as API keys and database credentials. \\
\hline
\textbf{express-rate-limit} & Provides rate limiting to protect the API from abuse by limiting the number of requests from a single IP address. \\
\hline
\textbf{jsonwebtoken} and \textbf{bcrypt} & Facilitate secure authentication and encryption. JWTs are used for managing user sessions, while bcrypt is used for hashing passwords. \\
\hline
\textbf{mongoose} and \textbf{mongodb} & Mongoose is an ODM (Object Data Modeling) library that provides a schema-based solution to model our application data. MongoDB serves as the database for storing application data. \\
\hline
\textbf{multer} & Handles file uploads in our API requests, supporting various file types such as PDFs and images. \\
\hline
\textbf{cors} & Manages Cross-Origin Resource Sharing (CORS) to allow or restrict resource access from different domains. \\
\hline
\end{tabular}
\caption{Dependencies and Their Roles}
\label{table:dependencies}
\end{table}


\begin{figure}[h]
\centering
\includegraphics[width=0.8cm]{mern_stack_architecture.png}
\caption{Architecture of the MERN Stack Application}
\label{fig:mern_architecture}
\end{figure}

\subsection{Frontend Development with React.js}

On the frontend, \textbf{React.js} was employed to build a dynamic and responsive user interface. \textbf{React.js} is a JavaScript library for building user interfaces, known for its component-based architecture and efficient rendering through a virtual DOM. 

To manage forms and validation, we integrated \textbf{React Hook Form} with \textbf{Zod} for schema validation. \textbf{React Hook Form} simplifies form handling by reducing the need for boilerplate code and improving performance, while \textbf{Zod} provides a robust validation schema to ensure data integrity and enforce constraints. 

\begin{figure}[h]
\centering
\includegraphics[width=0.8cm]{react_component_structure.png}
\caption{Component Structure in React Application}
\label{fig:react_components}
\end{figure}

The user interface was further enhanced using \textbf{shadcn} for UI components and \textbf{Framer Motion} for animations. \textbf{shadcn} offers a set of accessible and customizable UI components, while \textbf{Framer Motion} adds sophisticated animations, creating a visually appealing and interactive experience for users.

\subsection{Database Design and Management}

The application leverages \textbf{MongoDB}, a NoSQL database known for its flexibility and scalability. MongoDB's document-oriented data model allows for dynamic schema design, making it an ideal choice for applications with evolving data requirements.

For efficient querying and data aggregation, we employed the \textbf{MongoDB Aggregation Pipeline}. This feature enables complex data processing tasks, such as filtering, sorting, and grouping, to be performed directly within the database, optimizing performance and reducing the load on the application server.

\begin{table}[h]
\centering
\begin{tabular}{|l|l|}
\hline
\textbf{Feature} & \textbf{Description} \\
\hline
\textbf{Data Model} & Document-oriented \\
\textbf{Aggregation Pipeline} & Allows for complex data processing \\
\textbf{Scalability} & High scalability and performance \\
\hline
\end{tabular}
\caption{Database Design Features}
\label{tab:database_features}
\end{table}

\subsection{Deployment and Hosting}

For deployment and hosting, we utilized \textbf{Render.com} for backend services and \textbf{Vercel.com} for frontend hosting. \textbf{Render.com} offers a streamlined platform for deploying web applications and APIs, providing automatic scaling and continuous deployment features. \textbf{Vercel.com} specializes in frontend deployment, ensuring fast and reliable delivery of static assets and dynamic content.

We used \textbf{Google Cloud Storage Buckets} for storing and managing user-uploaded resumes and other files. The integration with \textbf{Google Cloud} provided a scalable and secure solution for file storage, ensuring high availability and durability.

\begin{figure}[h]
\centering
\includegraphics[width=0.8cm]{deployment_architecture.png}
\caption{Deployment Architecture of the Application}
\label{fig:deployment_architecture}
\end{figure}

In addition to the deployment platforms, we followed best practices for \textbf{API abuse management} using \textbf{Helmet}, a middleware for securing HTTP headers in Express applications. \textbf{Helmet} helps mitigate various security vulnerabilities by setting appropriate HTTP headers.

Credentials for various services were securely stored in a \texttt{.env} file, adhering to best practices for environment variable management and ensuring sensitive information is not exposed.

\subsection{Code Snippets}

Here is an example of how we used \textbf{JWT} for session management in our \textbf{Node.js} application:

\begin{verbatim}
const jwt = require('jsonwebtoken');

const generateToken = (user) => {
    return jwt.sign({ id: user._id }, process.env.JWT_SECRET, { expiresIn: '1h' });
};

const authenticateToken = (req, res, next) => {
    const token = req.headers['authorization'];
    if (token == null) return res.sendStatus(401);

    jwt.verify(token, process.env.JWT_SECRET, (err, user) => {
        if (err) return res.sendStatus(403);
        req.user = user;
        next();
    });
};
\end{verbatim}

This snippet demonstrates how tokens are generated and verified, ensuring secure user authentication and session management.

\begin{verbatim}
const express = require('express');
const app = express();

app.use(helmet());

app.get('/api/data', authenticateToken, (req, res) => {
    res.json({ message: 'Secure data' });
});
\end{verbatim}

In this snippet, \textbf{Helmet} is used to enhance API security by setting various HTTP headers.




