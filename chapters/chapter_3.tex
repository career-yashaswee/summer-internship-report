\section{Chapter 3 - Project Management and Development Lifecycle}
\subsection{Introduction to Project Management}
Project management is the backbone of successful software development. It ensures that the project is delivered on time, within scope, and meets the expected quality standards. Effective project management involves coordinating resources, tasks, and timelines to achieve the project goals efficiently.

\subsection{Requirement Gathering and Analysis}
Requirement gathering is the first and most crucial phase of the project. During this phase, we engaged with stakeholders, including mentors and potential users, to understand the specific needs and expectations from the software. We documented these requirements clearly and conducted a thorough analysis to ensure that every aspect was covered and aligned with the project’s objectives.

\begin{table}[h!]
\centering
\begin{tabular}{|l|p{10cm}|}
\hline
\textbf{Profession} & \textbf{Use Case} \\ \hline
Recent Graduate & Can use it to practice common behavioral interview questions. \\ \hline
Professional Seeking Career Change & Can utilize it to rehearse for technical interviews specific to their new target role. \\ \hline
Experienced Individual & Can conduct mock interviews on it to refine their overall interview delivery and build confidence before a high-stakes interview. \\ \hline
\end{tabular}
\caption{Use Cases Based on Profession}
\end{table}

\begin{table}[h!]
\centering
\begin{tabular}{|l|p{10cm}|}
\hline
\textbf{User Characteristic} & \textbf{Description} \\ \hline
\textbf{Experience Level} & Users will have varying levels of experience, ranging from recent graduates to seasoned professionals. \\ \hline
\textbf{Comfort with Interviews} & Comfort levels with interviews may vary, with some users being very confident while others may feel anxious or unsure. \\ \hline
\textbf{Technical Literacy} & Technical literacy is required for navigating the web application, including understanding how to use online tools and interfaces. \\ \hline
\textbf{Learning Pace} & Users may have different learning paces; some may need more time to understand the feedback provided, while others may progress quickly. \\ \hline
\textbf{Language Proficiency} & Proficiency in the language used in the application can affect how well users understand the interview questions and feedback. \\ \hline
\textbf{Goal Orientation} & Users may have different goals, such as improving interview skills for a specific job role or general preparation for future opportunities. \\ \hline
\end{tabular}
\caption{User Characteristics and Descriptions}
\label{table:user_characteristics}
\end{table}


\subsection{Planning and Task Allocation}
Planning involves breaking down the project into manageable tasks and allocating these tasks to team members based on their expertise. We used an iterative development approach, dividing the project into weekly sprints. Each team member was responsible for a specific domain, ensuring that all aspects of the project were covered comprehensively.

\begin{table}[h!]
\centering
\begin{tabular}{|c|l|l|}
\hline
\textbf{SNo.} & \textbf{Name}           & \textbf{Domain}            \\ \hline
1             & Yashaswee Kesharwani    & Project Integration        \\ \hline
2             & Nainshi Verma           & AI Integration             \\ \hline
3             & Satyam                  & Backend Development        \\ \hline
4             & Vikash Yadav            & Frontend Development       \\ \hline
5             & Madhav Sharma           & Database Development       \\ \hline
\end{tabular}
\caption{Team Members and Their Domains}
\end{table}

\subsection{Iterative Development and Sprints}
We adopted an iterative development model, which allowed for continuous improvement and integration of feedback. Each week, we completed a sprint, delivering a specific set of features. This approach helped us stay on track and make necessary adjustments in response to challenges encountered during development.

\subsection{Monitoring and Reporting Progress}
To ensure transparency and accountability, we held daily scrum meetings with our mentors. During these meetings, each team member reported on their progress, challenges, and planned tasks for the day. This consistent communication helped us identify and address issues quickly, keeping the project on schedule.

\subsection{Risk Management and Mitigation}
Risk management was an integral part of our project management strategy. We identified potential risks early in the project and developed mitigation plans to address them. This proactive approach allowed us to minimize the impact of unexpected challenges and ensured the smooth progression of the project.

\begin{table}[h!]
\centering
\begin{tabular}{|l|l|l|}
\hline
\textbf{Risk Identified}                             & \textbf{Mitigation Plan}                                                \\ \hline
Exorbitant Cost for Testing Google Gemini API SDK    & Set usage limits, monitor API usage, and optimize test cases.           \\ \hline
Delays in Backend Development                        & Implement a buffer in the timeline and conduct parallel tasks.          \\ \hline
Technical Issues During Frontend Integration         & Conduct preliminary integration tests and have backup tools ready.      \\ \hline
Limited Access to AI Integration Resources           & Schedule access in advance, explore alternative resources.              \\ \hline
Data Security Breach During Database Development     & Employ encryption, secure access protocols, and regular audits.         \\ \hline
Unavailability of Team Members Due to Health Issues  & Cross-train team members and have a backup plan for critical tasks.     \\ \hline
\end{tabular}
\caption{Few Examples of Risk Management and Mitigation Strategies}
% \label{tab:risk_management}
\end{table}
